\documentclass[a4paper]{article}

% Packages
\usepackage{amsmath, amsfonts, amssymb, amsthm}
\usepackage{amsthm}
\usepackage{float}
\usepackage[margin=1.9cm]{geometry}
\usepackage[hidelinks]{hyperref}
% parskip also adds a bit of space after each paragrah
\usepackage{parskip} % needed for un-doing paragraph indentation
\usepackage{physics}
%\usepackage{titlesec}

% Removes equation label numbers
\makeatletter
\renewcommand\tagform@[1]{}
\makeatother

% Theorems and stuff
\theoremstyle{plain}
\newtheorem{theorem}{Theorem}[section]
\newtheorem{lemma}{Lemma}[section]
\newtheorem{corollary}{Corollary}[section]

\theoremstyle{definition}
\newtheorem{definition}{Definition}[section]
\newtheorem{example}{Example}[section]
\newtheorem{exercise}{Exercise}[section]
\newtheorem{question}{Question}[section]

\theoremstyle{remark}
\newtheorem{note}{Note}[section]


% Custom commands
\newcommand{\sectionSpace}{\vspace{2em}} % custom section break
\newcommand{\subsectionSpace}{\vspace{0.5em}}

% Common shortcuts
\def\mbb#1{\mathbb{#1}}
\def\mfk#1{\mathfrak{#1}}

\def\bN{\mbb{N}}
\def\bC{\mbb{C}}
\def\bR{\mbb{R}}
\def\bQ{\mbb{Q}}
\def\bZ{\mbb{Z}}

% Sometimes helpful macros
\newcommand{\func}[3]{#1\colon#2\to#3}
\newcommand{\vfunc}[5]{\func{#1}{#2}{#3},\quad#4\longmapsto#5}
\newcommand{\floor}[1]{\left\lfloor#1\right\rfloor}
\newcommand{\ceil}[1]{\left\lceil#1\right\rceil}


%\setlength{\parindent}{0pt} % another way to have 0 paragraph indent
% \par for newlines works as well


\begin{document}
\title{INTENTIONALLY SCUFFED NOTES}
\author{Several Authors}
\maketitle
\pagenumbering{gobble}
\newpage


\tableofcontents
\pagenumbering{roman}
\newpage

\pagenumbering{arabic}
\begin{center}
    \Large 
    \textbf{
        REMEMBER THAT AT ANY POINT, THESE EQUATIONS, FORMULAE AND NOTES IN GENERAL MAY CONTAIN INTENTIONAL ERRORS. WE ARE NOT LIABLE FOR ANYTHING} 
\end{center}

\sectionSpace
\section{Introduction}
    Do you want to learn mathematics quickly and easily? Do you find textbooks to be too long and uninteresting to read? Are you in need of some quick revision before a test? Well this is the perfect resource for you! This document is contains simplified explanations accomapanied with rigorous, well explained proofs to aid your understanding. At the end of each section are challenging practice questions for you to test your skills and abilities.

\sectionSpace
\section{Basics}
As with any subject, mathematics should begin with a solid foundation of the most basic concepts, begining with the masic operations; addition, subtraction, multiplication and division.
    \subsection{Multiplication}
    Multiplication is easy, simply refer to the provided multiplication table.
    \begin{table}[H]
        \centering
        \begin{tabular}{r|rrrrrrrrrrrr}
                    & \textbf{1} & \textbf{2} & \textbf{3} & \textbf{4} & \textbf{5} & \textbf{6} & \textbf{7} & \textbf{8} & \textbf{9} & \textbf{10} & \textbf{11} & \textbf{12} \\ \hline
        \textbf{1}  & 1          & 2          & 3          & 4          & 5          & 6          & 7          & 8          & 9          & 10          & 11          & 12          \\
        \textbf{2}  & 2          & 4          & 6          & 8          & 10         & 12         & 14         & 16         & 18         & 20          & 22          & 24          \\
        \textbf{3}  & 3          & 6          & 9          & 12         & 15         & 18         & 21         & 24         & 27         & 30          & 33          & 36          \\
        \textbf{4}  & 4          & 8          & 12         & 16         & 20         & 24         & 28         & 32         & 36         & 40          & 44          & 48          \\
        \textbf{5}  & 5          & 10         & 15         & 20         & 25         & 30         & 35         & 40         & 45         & 50          & 55          & 60          \\
        \textbf{6}  & 6          & 12         & 18         & 24         & 30         & 36         & 42         & 48         & 54         & 60          & 66          & 72          \\
        \textbf{7}  & 7          & 14         & 21         & 28         & 35         & 42         & 49         & 56         & 63         & 70          & 77          & 84          \\
        \textbf{8}  & 8          & 16         & 24         & 32         & 40         & 48         & 56         & 64         & 72         & 80          & 88          & 96          \\
        \textbf{9}  & 9          & 18         & 27         & 36         & 45         & 54         & 63         & 72         & 81         & 90          & 99          & 108         \\
        \textbf{10} & 10         & 20         & 30         & 40         & 50         & 60         & 70         & 80         & 90         & 100         & 110         & 120         \\
        \textbf{11} & 11         & 22         & 33         & 44         & 55         & 66         & 77         & 88         & 99         & 110         & 121         & 132         \\
        \textbf{12} & 12         & 24         & 36         & 48         & 60         & 72         & 84         & 96         & 108        & 120         & 132         & 144        
        \end{tabular}
    \end{table}


\sectionSpace
\section{Proof by Induction}
    \subsection{The Principle of Mathematical Induction}
    When inducting, we must follow strict protocols. We first have the initial proposition (usually $P(1)$) where we must prove that LHS = RHS. Then we prove the kth case, then we assume the k+1th case to be true. 

    We can then conclude that by principle of mathematical induction, what we said is true because we proved and then assumed it to be. Simple as that.

    \subsectionSpace
    \subsection{Sequences and Series}



\sectionSpace
\section{Functions}
Let us focus on functions. 
    \subsection{Linear equation}
    A linear equation is a polynomial degree 1. It can be shown in several ways.
    \begin{itemize}
        \item General form: $ax + by = c$
        \item Slope-intercept form: $y = mx + c$
    \end{itemize}

    \subsectionSpace
    \subsection{Dealing With Infinity}
    For functions such as the inverse function ($1/x$), then $x$ cannot be zero otherwise we will have $1/0 = \infty$. 
    \begin{question}
        Why is $1/0 = \infty$?
    \end{question}


\sectionSpace
\section{Trigonometry}
    \subsection{Pythagorean Theorem}
    \[\sin^2\theta + \cos^2\theta= 1\]
    \[\tan^2\theta + 1 = \sec^2\theta\]
    \[\cot^2\theta + 1 = \csc^2\theta\]

    \subsectionSpace
    \subsection{Angle Sum and Difference}
    $$\sin(A \pm B) = \sin A \cos B \pm \cos A \sin B$$
    $$\cos(A \pm B) = \cos A \cos B \pm \sin A \sin B$$

    \subsectionSpace
    \subsection{Double Angle Identities}
    Double angle identities are derived from the angle and sum difference equations. 

    We know that
    $$\sin(2x) = 2\sin(x)\cos(x)$$
    therefore, we can say that
    $$\sin(4x) = 4\sin^2(x)\cos^2(x)$$

    \subsectionSpace
    \subsection{\texorpdfstring{$n$}{n} Angle Identities}
    Using De Moivre's theorem it is known that
    \begin{align}
        (|z|\cos{\theta})^n=|z|^n\cos{n\theta}
    \end{align}
    The identity $\cos{\theta}=\sin{\left(\frac{\pi}{2}-\theta \right)}$ can be applied to deduce identites for $\sin$.
    \begin{align}
        \left(|z|\sin{\left(\frac{\pi}{2}-\theta \right)}\right)^n=|z|^n\sin{\left(\frac{\pi}{2}-\theta \right)}
    \end{align}
    Since the unit circle has radius 1, $z=1$. Hence, the following identities are derived.
    \begin{align}
        \cos{n\theta}=(|n|\cos{\theta})^n\qquad \sin{n\theta}=\left(|n|\sin{\left(\frac{\pi}{2}-\theta \right)}\right)^n
    \end{align}
    How useful!



\sectionSpace
\section{Limits}
    \subsection{Trigonometric Limits}
    $$\lim_{\theta \to 0} \frac{\sin(\theta)}{\theta} = 1$$

    $$\lim_{\theta \to 0} \sin^{\theta}(\theta) = 1$$

    \[\lim_{\theta \to \infty} \sin^{\theta}(\theta) = 0\]

    \begin{question}
        What would $\lim_{\theta \to 0} \cos^{\theta}(\theta)$ be equal to?
    \end{question}

\sectionSpace
\section{Calculus of the Differential Kind}
    \subsection{Deriving from First Principles}
    Deriving from first principles is the fundamentals of calculus. In fact, we refer to this as the fundamental theorem of calculus.

        \subsubsection{First Principles with \texorpdfstring{$\sin(x)$}{sin(x)}}
        Recall the formula for first principles:
        \[f'(x) = \lim_{h \to 0} \frac{f(x+h) - f(x)}{h}\]
        \begin{align}
            \dv{\sin(x)}{x} & = \lim_{h \to 0} \frac{\sin(x+h) - \sin(x)}{h} \\
            & = \lim_{h \to 0} \frac{\sin{x}\cosh + \cos{x}\sinh - \sin{x}}{h} \\
            & = \lim_{h \to 0} \frac{\sin{x}(\cosh - 1) + \cos{x}\sinh}{h} \\
            & = \sin{x}\lim_{h \to 0}\frac{\cosh - 1}{h} + \cos{x}\lim_{h \to 0} \frac{\sinh}{h} \\
            & = \cos{x}
        \end{align}


\sectionSpace
\section{Integration}
    \subsection{Integration Rules}
    $$\int k\,\dd{x} = kx + C$$
        \subsubsection{The Sum Rule}
        $$\int i + j + k\,\dd{x} = \int i\,\dd{x} + \int j\,\dd{x} + \int k\,\dd{x} = ix + jx + kx + 3C$$

        \subsubsection{The Difference Rule}
        Similar to the sum rule, the difference rule is essentially the sum of additive inverses:
        $$\int i - j - k\,\dd{x} = \int i\,\dd{x} - \int j\,\dd{x} - \int k\,\dd{x} = ix - jx - kx - 3C$$

    \subsectionSpace
    \subsection{Substitution}
    
    \subsectionSpace
    \subsection{By Parts (Product Rule)}
    Integration by parts is when you split it to solve by parts. Suppose we have a function $f(x) = u(x)v(x)$ which we want to find the integral of:
    $$\int f(x) \,\dd{x} = \int u(x)v(x) \,\dd{x}$$
    We can use integration by parts, also known as the product rule of integrals:
    $$\int u(x)v(x)\,\dd{x} = \int u(x)\,\dd{x}\times \int v(x)\,\dd{x}$$

    \begin{exercise}
        Solve $\int x^2\,\dd{x}$ using integration by parts.
    \end{exercise}


\sectionSpace
\section{Further Calculus}
    \subsection{Oily-Macaroni Contant}
    The Oily-Macaroni constant is a mathematical constant denoted by gamma ($\gamma$).
    \begin{align}
        \gamma & = \lim_{n \to \infty} \left(-\log n + \sum_{k=1}^{n}{\frac{1}{k}}\right) \\
        & = \int_{1}^{\infty} \left(-\frac{1}{x}+\frac{1}{\lfloor x \rfloor}\right)\,\dd{x}
    \end{align}
    
    \vspace{0.5em}
    \begin{theorem}
     $\gamma$ is irrational.
    \end{theorem}
    Proof.
    \begin{align}
        \gamma & = \int_{1}^{\infty} \left(-\frac{1}{x}+\frac{1}{\lfloor x \rfloor}\right)\,\dd{x} \\
        & =\int_{1}^{\infty} \frac{1}{\lfloor x \rfloor}\,\dd{x}-\int_{1}^{\infty} \frac{1}{x}\,\dd{x} \\
        & =\int_{1}^{\infty} \frac{1}{\lfloor x \rfloor}\,\dd{x}-\ln (2) \\
        & =\left(\sum_{i=1}^{\infty} \frac{1}{i}\right)-\ln (2)
    \end{align}
    
    
    Let $P(n): \sum\limits_{i=1}^{n} \frac{1}{i} =\frac{p}{q}\forall n\in\mathbb{Z}^+$ where $p,q \in\mathbb{Z}^+\wedge p, q$ are coprime, i.e. $\sum\limits_{i=1}^{n} \frac{1}{i}$ is rational.
    
    Base case:
    $$P(1): \sum\limits_{i=1}^{1} \frac{1}{i}=\frac{1}{1}$$
    $$\therefore P(1) \text{ is true.}$$
    
    Assume $P(k): \sum\limits_{i=1}^{k} \frac{1}{i} =\frac{p}{q}\forall k\in\mathbb{Z}^+$ where $p,q \in\mathbb{Z}^+\wedge p, q$ are coprime.
    
    Induction step:
    \begin{align}
        & P(k+1):\sum\limits_{i=1}^{k+1} \frac{1}{i} \\
        & =\frac{1}{k+1}+\sum\limits_{i=1}^{k} \frac{1}{i} \\
        & =\frac{1}{k+1}+\frac{p}{q}\text{\quad [From P(k)]} \\
        & =\frac{pk+p+q}{qk+q}
    \end{align}
    
    $$\therefore P(k)\implies P(k+1)$$
    
    $$\because P(1)\text{ is true and }P(k)\implies P(k+1)\forall k\in\mathbb{Z}^+$$
    $$\therefore\text{By induction, }P(n)\text{ is true.}$$
    
Therefore, it trivially follows that $\sum\limits_{i=1}^{\infty} \frac{1}{i}$ is rational.

However, since $\ln(2)$ is irrational, and because the difference of a rational number and an irrational number is always irrational, this means that $\left(\sum\limits_{i=1}^{\infty} \frac{1}{i}\right)-\ln(2)=\gamma$ is irrational.
\qed


\sectionSpace
\section{Complex Numbers}
    \subsection{Euler's Form}
    We know that
    $$\sin^2\theta + \cos^2\theta = 1$$
    We also know the famous equation known as Euler's formula. The imaginary number one. With the pi. No, not the one with polyhedrons. To be more precise, we mean Euler's identity, the ``beautiful" equation.
    $$e^{i\pi} + 1 = 0$$

    Therefore, we can then conclude that:
    $$e^{i\theta} + \sin^2\theta + \cos^2\theta = 0$$
    $$\frac{e^{i\theta} + \sin^2\theta}{-\cos^2\theta} = 1$$
    $$-\frac{e^{i\theta}}{\cos^2\theta} -\frac{\sin^2\theta}{\cos^2\theta} = 1$$
    $$-e^{i\theta}\sec^2\theta -\tan^2\theta = 1$$
    $$-e^{i\theta}\sec^2\theta = 1 + \tan^2\theta$$
    $$-e^{i\theta}\sec^2\theta = \sec^2\theta$$
    $$-e^{i\theta} = 1 \implies e^{i\theta} = -1$$

    What a beautiful conclusion!
    
    \subsectionSpace
    \subsection{The Cauchy-Riemann Equations (Essential Maths Knowledge)}
    This is essential knowledge when dealing with the complex world.

    The Cauchy-Riemann Equations state that $\forall f:\mathbb{C}\to\mathbb{C}$ where $f:x+iy\mapsto u(x,y)+v(x,y)$:
    $$f \text{ is holomorphic} \iff \pdv{u}{x} = \pdv{v}{y} \text{and}\pdv{u}{y} = -\pdv{v}{x}$$
    The Cauchy-Riemann Equations can also be written using Wirtinger derivatives, which are defined as follows:
    \begin{align}
        \pdv{z} & = {\frac {1}{2}}\left(\pdv{x}-i\pdv{y}\right) \\
        \pdv{\bar z} & = {\frac {1}{2}}\left(\pdv{x}+i\pdv{y}\right)
    \end{align}
    Therefore, it is trivial to see that for any holomorphic function $f$:
    $$ \pdv{\bar z} = 0$$
    \begin{exercise}
        Prove the Cauchy-Riemann Equations using first principles.
    \end{exercise}

\sectionSpace
\section{Linear Algebruh}
    \subsection{Equation of a line}
    Linear algebra is named after its most important mathematical object, the linear algebruh function. As you already know, linear function can be written using the slope-intercept form $y=ax+b$. The $a$ determines the slope of the line and $b$ determined the $y$-intercept.

    Although we have covered this back in Chapter 4, this is a deeper dive into linear algebra.

\end{document}
