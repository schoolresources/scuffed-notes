\documentclass[a4paper]{article}

% Packages
\usepackage{amsmath}
\usepackage{amsthm}
\usepackage[margin=1.9cm]{geometry}
% parskip also adds a bit of space after each paragrah
\usepackage{parskip} % needed for un-doing paragraph indentation
%\usepackage{titlesec}


% Removes equation label numbers
\makeatletter
\renewcommand\tagform@[1]{}
\makeatother

% Theorems and stuff
\theoremstyle{definition}
\newtheorem{definition}{Definition}
\newtheorem{example}{Example}
\newtheorem{question}{Question}


% Custom commands
\newcommand{\sectionSpace}{\vspace{1em}} % custom section break

%\setlength{\parindent}{0pt} % another way to have 0 paragraph indent
% \par for newlines works as well


\begin{document}
\title{INTENTIONALLY SCUFFED NOTES}
\author{Several Authors}
\maketitle
\newpage

%\normalsize % reset \Large{} text. Otherwise, put this into a frame


\begin{center}
    \Large 
    \textbf{
        REMEMBER THAT AT ANY POINT, THESE EQUATIONS, FORMULAE AND NOTES IN GENERAL MAY CONTAIN INTENTIONAL ERRORS. WE ARE NOT LIABLE FOR ANYTHING} 
\end{center}

\sectionSpace
\section{Introduction}
    Edit later.



\sectionSpace
\section{Proof by Induction}
    \subsection{The Principle of Mathematical Induction}
    When inducting, we must follow strict protocols. We first have the initial proposition (usually $P(1)$) where we must prove that LHS = RHS. Then we prove the kth case, then the k+1 case. 

    We then conclude that by principle of mathematical induction, what we said is true because we proved it to be. Simple as that.

    \subsection{Sequences and Series}



\sectionSpace
\section{Functions}
Let us focus on functions. 
    \subsection{Linear equation}
    A linear equation is a polynomial degree 1. It can be shown in several ways.
    \begin{itemize}
        \item General form: $ax + by = c$
        \item Slope-intercept form: $y = mx + c$
    \end{itemize}


    \subsection{Dealing With Infinity}
    For functions such as the inverse function (1/x), then x cannot be zero otherwise we will have $1/0 = \infty$. 
    \begin{question}
        Why is $1/0 = \infty$?
    \end{question}


\sectionSpace
\section{Trigonometry}
    \subsection{Angle Sum and Difference}
    $$\sin(A \pm B) = \sin A \cos B \pm \cos A \sin B$$
    $$\cos(A \pm B) = \cos A \cos B \pm \sin A \sin B$$

    \subsection{Double Angle Identities}
    Double angle identities are derived from the angle and sum difference equations. 

    We know that
    $$\sin(2x) = 2\sin(x)\cos(x)$$
    therefore, we can say that
    $$\sin(4x) = 4\sin^2(x)\cos^2(x)$$

    \subsection{$n$ Angle Identities}
    Using De Moivre's theorem it is known that
    \begin{align}
        (|z|\cos{\theta})^n=|z|^n\cos{n\theta}
    \end{align}
    The identity $\cos{\theta}=\sin{\left(\frac{\pi}{2}-\theta \right)}$ can be applied to deduce identites for $\sin$.
    \begin{align}
        \left(|z|\sin{\left(\frac{\pi}{2}-\theta \right)}\right)^n=|z|^n\sin{\left(\frac{\pi}{2}-\theta \right)}
    \end{align}
    Since the unit circle has radius 1, $z=1$. Hence, the following identities are derived.
    \begin{align}
        \cos{n\theta}=(|n|\cos{\theta})^n\qquad \sin{n\theta}=\left(|n|\sin{\left(\frac{\pi}{2}-\theta \right)}\right)^n
    \end{align}
    How useful!





\sectionSpace
\section{Calculus of the Differential Kind}

\sectionSpace
\section{Integration}
    \subsection{Substitution}
    \subsection{By Parts}


\sectionSpace
\section{Complex Numbers}
    \subsection{Euler's Form}
    We know that
    $$\sin^\theta + \cos^2\theta = 1$$
    We also know the famous equation known as Euler's formula. The imaginary number one. With the pi. No, not the one with polyhedrons. To be more precise, we mean Euler's identity. The "beautiful" equation.
    $$e^{i\pi} + 1 = 0$$

    Therefore, we can then conclude that:
    $$e^{i\theta} + \sin^2\theta + \cos^2\theta = 0$$
    $$\frac{e^{i\theta} + \sin^2\theta}{-\cos^2\theta} = 1$$
    $$-\frac{e^{i\theta}}{\cos^2\theta} -\frac{\sin^2\theta}{\cos^2\theta} = 1$$
    $$-e^{i\theta}\sec^2\theta -\tan^2\theta = 1$$
    $$-e^{i\theta}\sec^2\theta = 1 + \tan^2\theta$$
    $$-e^{i\theta}\sec^2\theta = \sec^2\theta$$
    $$-e^{i\theta} = 1 \Rightarrow e^{i\theta} = -1$$

    What a beautiful conclusion!

\sectionSpace
\section{Linear Algebruh}
    \subsection{Vectors}

\end{document}
